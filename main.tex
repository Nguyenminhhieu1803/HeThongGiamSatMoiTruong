%%%%%%%%%%%%%%%%%%%%%%%%%%%%%%%%%%%%%%%%%%%%%%%%%%%%%%%%%%%%%%%%%%%%%%%%%%%%%%%%%%%%%%%%%
% Báo cáo Hệ thống Giám sát Môi trường
% LaTeX Template
% Phiên bản 1.0 (tạo ra ngày 05/10/2022)
%
% Template này có thể được download tại:
% https://github.com/cpc1996/HUS-Dissertation-Template
%
% Phiên bản 1.0 được chỉnh sửa bởi:
% Công Phương Cao (congphuongcao@gmail.com)
% Nguyễn Cảnh Việt (vietncp@gmail.com)
%
% Template được tham khảo từ một phiên bản template của:
% Steve Gunn (http://users.ecs.soton.ac.uk/srg/softwaretools/document/templates/)
% Sunil Patel (http://www.sunilpatel.co.uk/thesis-template/)
%
%%%%%%%%%%%%%%%%%%%%%%%%%%%%%%%%%%%%%%%%%%%%%%%%%%%%%%%%%%%%%%%%%%%%%%%%%%%%%%%%%%%%%%%%%


%----------------------------------------------------------------------------------------
%	(KHÔNG CHỈNH SỬA PHẦN NÀY)
%
%	PHẦN 1: CÁC PACKAGE CƠ BẢN VÀ CÁC TÙY CHỈNH VĂN BẢN
%----------------------------------------------------------------------------------------

\documentclass[
12pt,
oneside,
english,
doublespacing,
nolistspacing,
liststotoc,
parskip,
headsepline,
chapterinoneline,
]{HUSdissertation}

% \usepackage{siunitx} % Đã chú thích gói này để khắc phục lỗi \SI và \si

% Cấu hình tiếng Việt và Font
\usepackage{fontspec} % Gói để sử dụng font hệ thống
\setmainfont{Times New Roman} % Đặt font chính là Times New Roman
\setsansfont{Arial} % Tùy chọn: đặt font sans-serif là Arial
\setmonofont{Consolas} % Tùy chọn: đặt font monospace (cho code) là Consolas hoặc Courier New

% % Các gói font cũ hơn (đã chú thích để tránh xung đột với fontspec và tiếng Việt)
% \usepackage[utf8]{inputenc} % Đã chú thích để tránh cảnh báo khi dùng XeLaTeX/LuaLaTeX
% \usepackage[utf8]{vietnam}  % Đã chú thích để tránh xung đột và dùng fontspec/babel
% %\usepackage[T1]{fontenc}
% \usepackage{mathptmx} % Đã chú thích để tránh xung đột với Times New Roman từ fontspec

\usepackage{amsmath} % Hỗ trợ công thức toán học
\allowdisplaybreaks

% https://www.overleaf.com/learn/latex/Biblatex_citation_styles
\usepackage[backend=biber,style=numeric,citestyle=ieee,natbib=true]{biblatex} % Sử dụng biber cho unicode và tính năng hiện đại

\addbibresource{main.bib}

\usepackage[autostyle=true]{csquotes}


%----------------------------------------------------------------------------------------
%	PHẦN 2: CÁC PACKAGE BỔ TRỢ THÊM VÀO TRONG QUÁ TRÌNH BIÊN SOẠN
%----------------------------------------------------------------------------------------

\RequirePackage{setlst}		% Liệt kê/trích dẫn code

\usepackage{multirow}
\usepackage{subfigure}
\usepackage[fontsize=13pt]{scrextend}

%https://www.sascha-frank.com/latex-font-size.html
%https://tex.stackexchange.com/questions/103286/how-to-change-section-subsection-font-size
\usepackage{titlesec}

\titleformat{\section}
{\normalfont\fontsize{13}{13}\bfseries}{\thesection}{1em}{}

\titleformat{\subsection}
{\normalfont\fontsize{13}{13}\bfseries\itshape}{\thesubsection}{1em}{}

%https://tex.stackexchange.com/questions/351961/how-to-indent-code-in-beginverbatim
\usepackage{fancyvrb} 		% Fancy Verbatim
\fvset{tabsize=4,vspace=0pt,fontsize=\footnotesize}

\usepackage{longtable} 		% Bảng dài - Long table

\usepackage[figuresright]{rotating} % Bảng ngang - Sideways table 
\usepackage{tabularx}

\usepackage{fontawesome5} 	% Các biểu tượng, ký hiệu đặc biệt

\usepackage{tikz} 			% Vẽ hình
\usetikzlibrary{calc}

\usepackage{indentfirst}	% Lùi đầu dòng ở đầu đoạn văn
\setlength{\parindent}{0.5cm}

% Định nghĩa một số lệnh cần thiết để điều chỉnh định dạng cho một số nội dung nhất định
% (Đã di chuyển từ Chapter1.tex để toàn bộ tài liệu có thể sử dụng)
\newcommand{\keyword}[1]{\textbf{#1}}
\newcommand{\tabhead}[1]{\textbf{#1}}
\newcommand{\code}[1]{\texttt{#1}}
\newcommand{\file}[1]{\texttt{\bfseries#1}}
\newcommand{\option}[1]{\texttt{\itshape#1}}

%----------------------------------------------------------------------------------------
%	PHẦN 3: THÔNG TIN VỀ BÁO CÁO HỆ THỐNG GIÁM SÁT MÔI TRƯỜNG
%----------------------------------------------------------------------------------------

\author{Nguyễn Minh Hiếu} % Tên của bạn hoặc nhóm thực hiện
\thesistitle{HỆ THỐNG GIÁM SÁT CHẤT LƯỢNG MÔI TRƯỜNG THÔNG MINH SỬ DỤNG CÔNG NGHỆ IOT} % Tên báo cáo của bạn

\supervisor {PGS. TS. Nguyễn Bách Khoa} % Giảng viên hướng dẫn 1
\supervisorr{} % Để trống nếu không có GVHD 2

% Khai báo mới cho "Chuyên ngành" và "Khóa"
\NewDocumentCommand{\specialty}{m}{\newcommand{\specialtyname}{#1}}
\NewDocumentCommand{\course}{m}{\newcommand{\coursename}{#1}}

\field{Ngành Kĩ thuật Điện tử và Tin học} % Tên ngành đào tạo
\program{Chương trình đào tạo Chuẩn} % Tên chương trình đào tạo
\doctype{ĐỒ ÁN CHUYÊN NGÀNH} % Đã thay đổi thành "ĐỒ ÁN CHUYÊN NGÀNH"
% \newcommand{\docname}{\doctype} % Đã xóa dòng này để tránh lỗi "Command \docname already defined."

% Điền thông tin cho chuyên ngành và khóa
\specialty{Khoa học dữ liệu} % Tên chuyên ngành của bạn (ví dụ từ ảnh mẫu)
\course{2020} % Khóa học của bạn (ví dụ từ ảnh mẫu)

\university{\href{https://www.vnu.edu.vn/home/}{Trường Đại học Khoa học Tự nhiên}}
\department{\href{http://hus.vnu.edu.vn/gioi-thieu/co-cau-to-chuc/khoa-truc-thuoc/khoa-vat-ly.html}{Khoa Vật Lí}} % Thay đổi thành khoa của bạn

\AtBeginDocument{
	\hypersetup{pdftitle=\ttitle}
	\hypersetup{pdfauthor=\authorname}
}


\begin{document}
	
	\lstset{style=codeC}	% Thiết lập ngôn ngữ C/C++ là ngôn ngữ mặc định cho phần liệt kê souce code 
	
	\frontmatter 			% Sử dụng hệ thống đánh số La Mã (i, ii, iii, iv...) cho những trang trước phần mục lục
	
	\pagestyle{plain} 
	
	
	%----------------------------------------------------------------------------------------
	%	(KHÔNG CHỈNH SỬA PHẦN NÀY)
	%
	%	PHẦN 4: TRANG TIÊU ĐỀ/TRANG BÌA (TITLE PAGE)
	%----------------------------------------------------------------------------------------
	
	% TRANG BÌA CHÍNH:
	%----------------------------------------------------------------------------------------
%	(KHÔNG CHỈNH SỬA PHẦN NÀY)
%
%	Phần 6: TRANG TIÊU ĐỀ/TRANG BÌA (TITLE PAGE)
%----------------------------------------------------------------------------------------

% TRANG BÌA CHÍNH:
\begin{titlepage}
	\begin{tikzpicture}[overlay,remember picture]
		\draw [line width=3pt]
		($ (current page.north west) + (2.0cm,-2.0cm) $)
		rectangle
		($ (current page.south east) + (-1.5cm,1.8cm) $);
		\draw [line width=1pt]
		($ (current page.north west) + (2.15cm,-2.15cm) $)
		rectangle
		($ (current page.south east) + (-1.65cm,1.95cm) $);
	\end{tikzpicture}
	\begin{center}
		
		\vspace{0.5cm} % Điều chỉnh khoảng cách để tránh chạm viền
		\noindent%
		\large \, \href{https://www.vnu.edu.vn/home/}{ĐẠI HỌC QUỐC GIA HÀ NỘI}\\
		\vspace{-0.40cm}{\large \MakeUppercase \univname}\\
		\vspace{-0.40cm}{\large \bfseries \MakeUppercase \deptname}\\[0.1cm]
		\includegraphics[width=0.25\textwidth]{Logo/Logo_HUS_notext_nocolor} % Logo cho bìa chính
		
		\vspace{0.3cm} % Khoảng cách
		
		% Dòng "ĐỒ ÁN CHUYÊN NGÀNH" (tức là nội dung của \doctype)
		{\large \textbf{\docname}\par}
		
		\vspace{0.8cm} % Khoảng cách sau "ĐỒ ÁN CHUYÊN NGÀNH" và trước tiêu đề chính
		
		{\Large \bfseries \MakeUppercase\ttitle\par} % Tiêu đề báo cáo chính
		
		\vspace{0.8cm} % Khoảng cách sau tiêu đề báo cáo và trước các thông tin chi tiết
		
		% Sử dụng tabular để căn chỉnh các dòng thông tin chi tiết (Giống hệt sub-cover.tex)
		\begin{tabular}{r @{\hspace{0.2cm}} l}
			\normalsize Giảng viên hướng dẫn & \normalsize : \supname \\
			\normalsize Sinh viên thực hiện 1 & \normalsize : \authorname \\
			\normalsize MSSV 1 & \normalsize : [Điền MSSV của bạn] \\ % <--- ĐIỀN MSSV CỦA BẠN VÀO ĐÂY
			% Nếu có sinh viên 2 (supnamee đã dùng cho GVHD 2, cần đổi tên biến nếu muốn)
			% \normalsize Sinh viên thực hiện 2 & \normalsize : \supnamee \\
			% \normalsize MSSV 2 & \normalsize : [Điền MSSV của sinh viên 2] \\
			\normalsize Chuyên ngành & \normalsize : \specialtyname \\
			\normalsize Khóa & \normalsize : \coursename \\
			\normalsize Chương trình đào tạo & \normalsize : \progname \\
		\end{tabular}
		
		\vfill % Giữ nguyên vfill
		
		\par % Đảm bảo dòng dưới là một đoạn văn bản riêng, tránh bị dính với minipage
		{\normalsize \textbf{Hà Nội - \the\year{}}} % Giữ nguyên normalsize hoặc có thể thử \small nếu vẫn tràn
	\end{center}
\end{titlepage}
	
	% TRANG BÌA PHỤ:
	\include{sub-cover}
	
	
	%----------------------------------------------------------------------------------------
	%	PHẦN 5: DANH NGÔN (QUOTES)
	%----------------------------------------------------------------------------------------
	
	\vspace*{0.2\textheight}
	
	\noindent\enquote{\itshape 
		Cái tôi và sự hiểu biết tỷ lệ nghịch với nhau. Hiểu biết càng nhiều cái tôi càng bé. Hiểu biết càng ít, cái tôi càng to.
	}\bigbreak
	
	\hfill Albert Einstein
	
	
	%----------------------------------------------------------------------------------------
	%	PHẦN 6: LỜI CẢM ƠN (ACKNOWLEDGEMENTS)
	%----------------------------------------------------------------------------------------
	
	\begin{acknowledgements}
		\addchaptertocentry{\acknowledgementname}
		\thispagestyle{empty}
		Lời cảm ơn chân thành đến thầy/cô hướng dẫn, gia đình, bạn bè và những người đã hỗ trợ tôi trong quá trình thực hiện báo cáo này. Đặc biệt, tôi xin bày tỏ lòng biết ơn sâu sắc đến [Tên Giảng viên Hướng dẫn] đã tận tình chỉ bảo và truyền đạt kiến thức quý báu, giúp tôi hoàn thành nghiên cứu này.
	\end{acknowledgements}
	
	
	%----------------------------------------------------------------------------------------
	%	(KHÔNG CHỈNH SỬA PHẦN NÀY)
	%
	%	PHẦN 7: MỤC LỤC (LIST OF CONTENTS/FIGURES/TABLES PAGES)
	%----------------------------------------------------------------------------------------
	
	\begin{spacing}{1.15}
		\tableofcontents 	% In ra mục lục chính
	\end{spacing}
	
	\begin{spacing}{1.15}
		\listoffigures 		% In ra danh sách hình vẽ
	\end{spacing}
	
	\begin{spacing}{1.15}
		\listoftables		% In ra danh sách bảng
	\end{spacing}
	
	
	%----------------------------------------------------------------------------------------
	%	PHẦN 8: DANH SÁCH TÊN VIẾT TẮT (ABBREVIATIONS)
	%----------------------------------------------------------------------------------------
	
	\begin{abbreviations}{ll} % Thêm danh sách tên viết tắt (dưới dạng một bảng có 2 cột)
		
		\textbf{IoT} & \textbf{I}nternet \textbf{o}f \textbf{T}hings \\
		\textbf{MCU} & \textbf{M}icro\textbf{c}ontroller \textbf{U}nit \\
		\textbf{API} & \textbf{A}pplication \textbf{P}rogramming \textbf{I}nterface \\
		\textbf{PM2.5} & \textbf{P}articulate \textbf{M}atter 2.5 \\
		\textbf{AQI} & \textbf{A}ir \textbf{Q}uality \textbf{I}ndex \\
		\textbf{DHT} & \textbf{D}igital \textbf{H}umidity and \textbf{T}emperature \\
		\textbf{NTP} & \textbf{N}etwork \textbf{T}ime \textbf{P}rotocol \\
		\textbf{HTTP} & \textbf{H}yper\textbf{T}ext \textbf{T}ransfer \textbf{P}rotocol \\
		\textbf{JSON} & \textbf{J}ava\textbf{S}cript \textbf{O}bject \textbf{N}otation \\
		\textbf{PHP} & \textbf{P}HP: \textbf{H}ypertext \textbf{P}reprocessor \\
		\textbf{MySQL} & \textbf{M}y\textbf{S}tructured \textbf{Q}uery \textbf{L}anguage \\
		\textbf{CSS} & \textbf{C}ascading \textbf{S}tyle \textbf{S}heets \\
		\textbf{JS} & \textbf{J}ava\textbf{S}cript \\
		\textbf{XAMPP} & \textbf{X}-platform \textbf{A}pache \textbf{M}ySQL \textbf{P}HP \textbf{P}erl \\
		
	\end{abbreviations}
	
	
	% Trong PHẦN 9: CÁC THÔNG SỐ KỸ THUẬT QUAN TRỌNG
	\begin{constants}{lr@{${}={}$}l}
		Điện áp hoạt động & $V_{op}$ & 5 V \\
		Dải nhiệt độ DHT11 & $T_{range}$ & 0 đến 50 \textdegree C \\
		Dải độ ẩm DHT11 & $H_{range}$ & 20\% đến 90\% \\
		Tần suất cập nhật mặc định & $f_{update}$ & 30 giây \\
	\end{constants}
	
	% Trong PHẦN 10: DANH SÁCH KÝ HIỆU
	\begin{symbols}{lll}
		$T$		& Nhiệt độ		& \textdegree C \\
		$H$		& Độ ẩm			& \% \\
		$V_{in}$ & Điện áp đầu vào & V \\
		$I_{out}$ & Dòng điện đầu ra & A \\
		$t_{stamp}$ & Thời gian cập nhật & giây \\
	\end{symbols}
	
	
	%----------------------------------------------------------------------------------------
	%	(KHÔNG CHỈNH SỬA PHẦN NÀY)
	%
	%	PHẦN 11: LỜI ĐỀ TẶNG (DEDICATION)
	%----------------------------------------------------------------------------------------
	
	%\dedicatory{Dành tặng/Dành cho/Gửi tới\ldots} 
	
	
	%----------------------------------------------------------------------------------------
	%	PHẦN 12: NỘI DUNG/CÁC CHƯƠNG BÁO CÁO (THESIS CONTENT - CHAPTERS)
	%----------------------------------------------------------------------------------------
	
	\mainmatter % Bắt đầu đánh số trang (1,2,3...)
	
	\pagestyle{plain}
	
	% Hãy thêm những chương (chapter) của báo cáo vào thư mục Chapters
	% Hãy bỏ chú thích những dòng nếu bạn đã bổ sung những chương vào
	
	% MỞ ĐẦU (LỜI MỞ ĐẦU - Chương 0)

\chapter*{MỞ ĐẦU} % Tên của chương
\addcontentsline{toc}{chapter}{MỞ ĐẦU} % Thêm tên chương vào mục lục

\label{Chapter0} % Để trích dẫn chương này ở chỗ nào đó trong bài, hãy sử dụng lệnh \ref{Chapter0} 

%----------------------------------------------------------------------------------------

Phần "MỞ ĐẦU" của báo cáo này trình bày tổng quan về hệ thống giám sát nhiệt độ và độ ẩm thông minh, một giải pháp được phát triển nhằm theo dõi chất lượng không khí trong môi trường nhà ở và văn phòng. Hệ thống này được thiết kế với mục tiêu cung cấp dữ liệu môi trường theo thời gian thực, hỗ trợ người dùng theo dõi lịch sử dữ liệu, và đưa ra cảnh báo kịp thời khi các thông số vượt ngưỡng cho phép.

Hệ thống nổi bật với khả năng triển khai đơn giản, chi phí thấp, và giao diện web thân thiện, dễ sử dụng, phù hợp cho các ứng dụng giám sát chất lượng không khí trong nhà. Báo cáo này sẽ đi sâu vào các khía cạnh từ bối cảnh và sự cần thiết, mục tiêu, đến thiết kế phần cứng và phần mềm, quy trình triển khai, cũng như kết quả hoạt động, các thách thức đã gặp phải và hướng phát triển trong tương lai.

Bố cục của báo cáo được tổ chức như sau:
\begin{itemize}
	\item \textbf{Chương 1: Tổng quan về hệ thống giám sát nhiệt độ và độ ẩm} trình bày bối cảnh, sự cần thiết và các thành phần chính của hệ thống.
	\item \textbf{Chương 2: Thiết kế và triển khai hệ thống} mô tả chi tiết kiến trúc phần cứng và phần mềm, cùng với quy trình xây dựng và lắp đặt hệ thống.
	\item \textbf{Chương 3: Kết quả và đánh giá hệ thống} trình bày các dữ liệu thu thập được, phân tích hiệu quả hoạt động và các tính năng cảnh báo, điều khiển.
	\item \textbf{Chương 4: Kết luận và định hướng phát triển} tóm tắt những thành quả chính và đề xuất các hướng cải tiến, mở rộng hệ thống trong tương lai.
	\item \textbf{Phụ lục} bao gồm mã nguồn chi tiết của các module hệ thống.
\end{itemize}
	% Chương 1

\chapter{TỔNG QUAN VỀ HỆ THỐNG GIÁM SÁT NHIỆT ĐỘ VÀ ĐỘ ẨM} % Tên của chương

\label{Chapter1} % Để trích dẫn chương này ở chỗ nào đó trong bài, hãy sử dụng lệnh \ref{Chapter1} 

%----------------------------------------------------------------------------------------

\section{Bối cảnh và sự cần thiết của giám sát môi trường trong nhà}
Chất lượng không khí trong nhà đóng vai trò vô cùng quan trọng đối với sức khỏe và năng suất làm việc của con người. Con người dành phần lớn thời gian trong nhà, nơi chất lượng không khí có thể bị ảnh hưởng bởi nhiều yếu tố như nhiệt độ, độ ẩm, và các chất ô nhiễm. Nhiệt độ và độ ẩm không phù hợp không chỉ gây khó chịu mà còn tạo điều kiện cho vi khuẩn, nấm mốc phát triển, ảnh hưởng tiêu cực đến hệ hô hấp và sức khỏe tổng thể. Do đó, việc giám sát liên tục và chính xác các thông số này là cần thiết để duy trì một môi trường sống và làm việc lành mạnh. Hệ thống giám sát tự động, đặc biệt là ứng dụng công nghệ IoT, mang đến giải pháp hiệu quả để thu thập dữ liệu, phân tích xu hướng và kịp thời đưa ra cảnh báo khi các chỉ số vượt ngưỡng an toàn.

\section{Mục tiêu của hệ thống}
Mục tiêu tổng quát của dự án là thiết kế và triển khai một hệ thống giám sát nhiệt độ và độ ẩm trong nhà một cách hiệu quả, dễ sử dụng và chi phí thấp, góp phần nâng cao chất lượng môi trường sống và làm việc.

Các mục tiêu cụ thể bao gồm:
\begin{itemize}
	\item Thu thập dữ liệu nhiệt độ và độ ẩm theo thời gian thực từ cảm biến \keyword{DHT11}.
	\item Xây dựng một giao diện web trực quan, cho phép người dùng dễ dàng theo dõi dữ liệu hiện tại, xem biểu đồ lịch sử và quản lý thiết bị.
	\item Cho phép người dùng thiết lập các ngưỡng cảnh báo cho nhiệt độ và độ ẩm, và hiển thị trạng thái cảnh báo trực tiếp trên giao diện.
	\item Cung cấp khả năng điều khiển bật/tắt chức năng đo của cảm biến thông qua giao diện web.
	\item Hỗ trợ xuất dữ liệu lịch sử ra định dạng CSV để tiện cho việc phân tích chuyên sâu.
\end{itemize}

\section{Các thành phần chính của hệ thống}
Hệ thống giám sát nhiệt độ và độ ẩm thông minh bao gồm các thành phần chính được phân loại thành phần cứng và phần mềm, cùng với luồng dữ liệu tương tác giữa chúng.

\subsection{Phần cứng}
\begin{itemize}
	\item \textbf{Cảm biến DHT11}: Là loại cảm biến chính được sử dụng để đo nhiệt độ và độ ẩm môi trường. Cảm biến này có ưu điểm về chi phí thấp và dễ tích hợp.
	\item \textbf{Bộ vi điều khiển ESP32}: Đóng vai trò là bộ não của hệ thống, \file{ESP32} chịu trách nhiệm đọc dữ liệu từ cảm biến \keyword{DHT11}, xử lý sơ bộ và truyền dữ liệu lên máy chủ thông qua kết nối Wi-Fi. \file{ESP32} cũng điều khiển các \keyword{LED} thông báo trạng thái hoạt động.
	\item \textbf{Module Wi-Fi}: Tích hợp sẵn trong \file{ESP32}, được sử dụng để thiết lập kết nối không dây với mạng cục bộ, cho phép truyền dữ liệu đến máy chủ \keyword{localhost}.
	\item \textbf{LED thông báo}:
	\begin{itemize}
		\item \textbf{LED hoạt động (màu xanh lá)}: Sáng liên tục khi cảm biến đang đo và gửi dữ liệu.
		\item \textbf{LED cảnh báo (màu đỏ)}: Sẽ bật sáng khi nhiệt độ hoặc độ ẩm vượt quá ngưỡng đã thiết lập.
	\end{itemize}
	\item \textbf{Nguồn điện Adapter 5V}: Cung cấp năng lượng cho toàn bộ hệ thống phần cứng hoạt động.
\end{itemize}

\subsection{Phần mềm}
\begin{itemize}
	\item \textbf{Firmware (Nhúng trên ESP32)}:
	\begin{itemize}
		\item Ngôn ngữ lập trình: \code{C/C++} sử dụng \code{Arduino IDE}.
		\item Chức năng: Đọc dữ liệu từ \keyword{DHT11}, đồng bộ thời gian thông qua \keyword{NTPClient} để gán \keyword{timestamp} chính xác cho dữ liệu, gửi yêu cầu \keyword{HTTP GET} đến \keyword{API} điều khiển, tạo gói dữ liệu \keyword{JSON} và gửi qua \keyword{HTTP POST} đến máy chủ. Firmware cũng điều khiển các \keyword{LED} thông báo trạng thái.
	\end{itemize}
	\item \textbf{Hệ thống Backend (Máy chủ cục bộ)}:
	\begin{itemize}
		\item Nền tảng: \code{XAMPP} (bao gồm \code{Apache} làm web server và \code{PHP} làm ngôn ngữ xử lý phía máy chủ).
		\item Cơ sở dữ liệu: \code{MySQL} được sử dụng để lưu trữ dữ liệu nhiệt độ, độ ẩm, \keyword{timestamp} và \keyword{device\_id} trong bảng \code{sensor\_readings} và \code{device\_settings}.
		\item Các \keyword{API} chính:
		\begin{itemize}
			\item \code{post\_data.php}: Nhận dữ liệu \keyword{JSON} từ \file{ESP32} và lưu vào \code{MySQL}.
			\item \code{get\_dht\_status.php}, \code{toggle\_dht\_status.php}: Quản lý trạng thái bật/tắt cảm biến.
			\item \code{save\_alert\_thresholds.php}, \code{get\_alert\_thresholds.php}: Quản lý các ngưỡng cảnh báo.
			\item \code{get\_devices.php}: Cung cấp danh sách các thiết bị.
			\item \code{get\_current\_data.php}, \code{get\_history\_data.php}: Cung cấp dữ liệu hiện tại và lịch sử.
			\item \code{export\_data\_csv.php}: Xuất dữ liệu lịch sử ra \file{CSV}.
		\end{itemize}
	\end{itemize}
	\item \textbf{Hệ thống Frontend (Giao diện người dùng web)}:
	\begin{itemize}
		\item Công nghệ: \code{HTML}, \code{CSS} và \code{JavaScript}.
		\item Chức năng: Hiển thị dữ liệu nhiệt độ và độ ẩm theo thời gian thực dưới dạng thẻ lớn, biểu đồ lịch sử tương tác (sử dụng \code{Chart.js} và \code{Luxon}), trang cài đặt ngưỡng cảnh báo, và khả năng điều khiển bật/tắt cảm biến. Giao diện cũng hiển thị thông tin kỹ thuật về cảm biến \keyword{DHT11}.
		\item Cảnh báo: Kích hoạt hiệu ứng nhấp nháy đỏ trên giao diện khi dữ liệu vượt ngưỡng.
	\end{itemize}
\end{itemize}
	%% Chương 2

\chapter{THIẾT KẾ VÀ TRIỂN KHAI HỆ THỐNG} % Tên của chương

\label{Chapter2} % Để trích dẫn chương này ở chỗ nào đó trong bài, hãy sử dụng lệnh \ref{Chapter2} 

%----------------------------------------------------------------------------------------

\section{Kiến trúc tổng thể của hệ thống}
Hệ thống giám sát nhiệt độ và độ ẩm thông minh được xây dựng theo kiến trúc IoT cơ bản, bao gồm các lớp cảm biến, xử lý dữ liệu tại biên, truyền thông, máy chủ và giao diện người dùng. Dữ liệu di chuyển tuần tự qua các lớp này để đảm bảo việc thu thập, xử lý và hiển thị thông tin môi trường một cách liền mạch.

\begin{figure}[hbtp]
	\centering
	\includegraphics[width=0.8\textwidth]{Figures/architectural_diagram.png} % Thay thế bằng sơ đồ khối thực tế của bạn
	\caption{Sơ đồ kiến trúc tổng thể của hệ thống giám sát nhiệt độ và độ ẩm}
	\label{fig:architecture}
\end{figure}

Sơ đồ kiến trúc tổng thể của hệ thống được minh họa trong Hình~\ref{fig:architecture}.
\begin{itemize}
	\item \textbf{Lớp cảm biến}: Bao gồm cảm biến \keyword{DHT11} để thu thập dữ liệu nhiệt độ và độ ẩm.
	\item \textbf{Lớp biên (Edge Layer)}: \file{ESP32} xử lý dữ liệu thô từ cảm biến, đồng bộ thời gian và điều khiển \keyword{LED}.
	\item \textbf{Lớp truyền thông}: \file{ESP32} sử dụng kết nối \keyword{Wi-Fi} để gửi dữ liệu lên máy chủ cục bộ.
	\item \textbf{Lớp máy chủ (Server Layer)}: Máy chủ \keyword{localhost} (\code{XAMPP}) với \code{PHP} và \code{MySQL} nhận, lưu trữ và xử lý dữ liệu, đồng thời cung cấp các \keyword{API} cho giao diện web.
	\item \textbf{Lớp giao diện người dùng}: Ứng dụng web (\code{HTML/CSS/JavaScript}) cung cấp giao diện trực quan cho người dùng theo dõi và tương tác với hệ thống.
\end{itemize}

\section{Thiết kế phần cứng}
\subsection{Sơ đồ mạch kết nối cảm biến và ESP32}
Phần cứng của hệ thống được thiết kế để đảm bảo sự nhỏ gọn và hiệu quả. Cảm biến \keyword{DHT11} được kết nối với chân \code{GPIO4} của \file{ESP32}. Ngoài ra, một \keyword{LED} màu xanh lá cây được kết nối với \code{GPIO15} để báo hiệu trạng thái hoạt động/đang đo, và một \keyword{LED} màu đỏ được kết nối với \code{GPIO13} để báo hiệu cảnh báo khi nhiệt độ hoặc độ ẩm vượt ngưỡng. Nguồn điện \code{5V} được cấp thông qua adapter.

\begin{figure}[hbtp]
	\centering
	\includegraphics[width=0.7\textwidth]{Figures/circuit_diagram.png} % Thay thế bằng sơ đồ mạch thực tế của bạn
	\caption{Sơ đồ mạch kết nối cảm biến DHT11 và ESP32}
	\label{fig:circuit}
\end{figure}
Sơ đồ mạch chi tiết được trình bày trong Hình~\ref{fig:circuit}.

\subsection{Lựa chọn linh kiện và lý do}
\begin{itemize}
	\item \textbf{Cảm biến DHT11}: Được lựa chọn vì tính phổ biến, chi phí thấp, và khả năng đo đồng thời nhiệt độ và độ ẩm, phù hợp cho các ứng dụng giám sát môi trường trong nhà cơ bản.
	\item \textbf{Vi điều khiển ESP32}: Là lựa chọn tối ưu nhờ tích hợp \keyword{Wi-Fi} và \keyword{Bluetooth}, khả năng xử lý mạnh mẽ, tiết kiệm năng lượng và cộng đồng hỗ trợ lớn. Điều này giúp đơn giản hóa việc truyền dữ liệu không dây và tích hợp các tính năng khác trong tương lai.
\end{itemize}

\section{Thiết kế phần mềm}
\subsection{Thiết kế Firmware trên ESP32}
Firmware cho \file{ESP32} được phát triển bằng ngôn ngữ \code{C/C++} trên nền tảng \code{Arduino IDE}.
\begin{itemize}
	\item \textbf{Thu thập dữ liệu}: Đọc giá trị nhiệt độ và độ ẩm từ \keyword{DHT11}.
	\item \textbf{Đồng bộ thời gian}: Sử dụng thư viện \code{NTPClient} để đồng bộ thời gian từ các máy chủ \code{NTP} công cộng (\code{vn.pool.ntp.org}), đảm bảo dữ liệu có \keyword{timestamp} chính xác.
	\item \textbf{Điều khiển hoạt động}: Định kỳ gửi yêu cầu \keyword{HTTP GET} đến \code{API} \file{api/get\_dht\_status.php} để kiểm tra trạng thái bật/tắt đo từ máy chủ. Nếu tính năng đo bị tắt, \file{ESP32} sẽ ngừng gửi dữ liệu và tắt \keyword{LED} thông báo.
	\item \textbf{Gửi dữ liệu}: Đóng gói dữ liệu nhiệt độ, độ ẩm, \keyword{device\_id} và \keyword{timestamp} vào định dạng \keyword{JSON} và gửi lên máy chủ qua \keyword{HTTP POST} đến \file{post\_data.php}.
	\item \textbf{Điều khiển LED}: Điều khiển \keyword{LED} xanh lá sáng khi đang đo và \keyword{LED} đỏ sáng khi các thông số vượt ngưỡng cảnh báo (ngưỡng được lấy từ \file{api/get\_alert\_thresholds.php}).
\end{itemize}

\subsection{Thiết kế Backend và Cơ sở dữ liệu}
Hệ thống \keyword{backend} được xây dựng trên môi trường \code{XAMPP} (bao gồm \code{Apache} làm web server và \code{MySQL} làm cơ sở dữ liệu) với ngôn ngữ \code{PHP}. Cấu trúc thư mục của hệ thống \keyword{backend} được tổ chức như sau:

\begin{itemize}
	\item \file{xampp/htdocs/TH\_DO\_AN/}: Chứa các file giao diện chính và các API cấp cao.
	\item \file{xampp/htdocs/TH\_DO\_AN/api/}: Chứa các API con liên quan đến quản lý dữ liệu và thiết bị.
	\item \file{xampp/htdocs/Login/}: Chứa các file liên quan đến chức năng đăng nhập/đăng ký.
	\item \file{xampp/htdocs/Login/api/users/}: Chứa các API liên quan đến quản lý người dùng.
\end{itemize}

\begin{itemize}
	\item \textbf{Cơ sở dữ liệu \code{MySQL}}:
	\begin{itemize}
		\item Bảng \code{sensor\_readings}: Lưu trữ các bản ghi dữ liệu nhiệt độ, độ ẩm, \keyword{timestamp} và \keyword{device\_id} từ \file{ESP32}.
		\item Bảng \code{device\_settings}: Lưu trữ các cài đặt của thiết bị như trạng thái bật/tắt đo (\code{dht\_enabled}) và các ngưỡng cảnh báo (\code{temp\_threshold}, \code{humidity\_threshold}).
		\item Các bảng khác liên quan đến người dùng (từ thư mục \file{Login}), ví dụ: bảng \code{users} để lưu thông tin đăng nhập.
	\end{itemize}
	\item \textbf{Các \keyword{API} \code{PHP} chính}:
	\begin{itemize}
		\item \file{post\_data.php} (trong \file{TH\_DO\_AN/}): Nhận dữ liệu \keyword{JSON} từ \file{ESP32} và chèn vào bảng \code{sensor\_readings}.
		\item \file{get\_current\_data.php} (trong \file{TH\_DO\_AN/}): Truy vấn và trả về dữ liệu mới nhất.
		\item \file{get\_history\_data.php} (trong \file{TH\_DO\_AN/}): Truy vấn và trả về dữ liệu lịch sử theo khoảng thời gian.
		\item \file{api/export\_data\_csv.php} (trong \file{TH\_DO\_AN/api/}) : Tạo và xuất file \file{CSV} từ dữ liệu lịch sử.
		\item \file{api/get\_dht\_status.php} (trong \file{TH\_DO\_AN/api/}): Trả về trạng thái bật/tắt đo cho \file{ESP32}.
		\item \file{api/toggle\_dht\_status.php} (trong \file{TH\_DO\_AN/api/}): Cập nhật trạng thái bật/tắt đo từ giao diện web.
		\item \file{api/save\_alert\_thresholds.php} (trong \file{TH\_DO\_AN/api/}): Lưu các ngưỡng cảnh báo.
		\item \file{api/get\_alert\_thresholds.php} (trong \file{TH\_DO\_AN/api/}): Trả về các ngưỡng cảnh báo đã lưu.
		\item \file{api/get\_devices.php} (trong \file{TH\_DO\_AN/api/}): Cung cấp danh sách các \keyword{device\_id}.
		\item \file{api/save\_email\_settings.php}, \file{api/save\_update\_frequency.php}, \file{api/get\_update\_frequency.php}: Các API cấu hình khác.
		\item Các API liên quan đến người dùng (trong \file{Login/api/users/}) như \file{login.php}, \file{register.php}, \file{check\_login.php}, \file{logout.php}.
	\end{itemize}
\end{itemize}

\subsection{Thiết kế Frontend và Giao diện người dùng}
Giao diện người dùng được xây dựng bằng \code{HTML}, \code{CSS} và \code{JavaScript}, chạy trên trình duyệt web, cung cấp một cái nhìn trực quan và khả năng tương tác với hệ thống. Cấu trúc file giao diện chính bao gồm \file{index.php}, \file{main.css}, \file{main.js}, và \file{dashboard\_template.html} trong thư mục \file{TH\_DO\_AN/}. Các file liên quan đến đăng nhập/đăng ký nằm trong thư mục \file{Login/}.

\begin{itemize}
	\item \textbf{Trang "Dữ liệu Hiện Tại"}: Hiển thị các giá trị nhiệt độ và độ ẩm mới nhất dưới dạng thẻ lớn, cùng với thời gian cập nhật gần nhất.
	\item \textbf{Trang "Lịch sử Dữ liệu"}: Cho phép người dùng chọn khoảng thời gian (ví dụ: 1 giờ) và hiển thị dữ liệu lịch sử dưới dạng biểu đồ đường tương tác, sử dụng thư viện \code{Chart.js} và \code{Luxon}.
	\item \textbf{Trang "Cảnh Báo"}: Cho phép người dùng nhập và lưu các ngưỡng cảnh báo cho nhiệt độ và độ ẩm. Trang này cũng hiển thị trạng thái hiện tại (Bình thường/Vượt ngưỡng) và kích hoạt hiệu ứng nhấp nháy đỏ trên giao diện khi có cảnh báo.
	\item \textbf{Trang "Thông tin thiết bị"}: Cung cấp thông tin chi tiết về cảm biến \keyword{DHT11} và cách nó hoạt động.
	\item \textbf{Chức năng đăng nhập/đăng ký}: Sử dụng \file{login\_register.html} và các file \code{CSS/JS} liên quan (\file{auth\_styles.css}, \file{auth\_script.js}) để quản lý người dùng.
	\item \textbf{Tính năng điều khiển}: Nút "Đang đo" trên giao diện cho phép gửi lệnh đến \code{API} \file{api/toggle\_dht\_status.php} để bật hoặc tắt chức năng đo của cảm biến \file{ESP32}.
\end{itemize}

\section{Quy trình triển khai}
\begin{itemize}
	\item \textbf{Cài đặt môi trường máy chủ}: Cài đặt \code{XAMPP} trên máy tính cục bộ để thiết lập môi trường \code{Apache}, \code{PHP} và \code{MySQL}.
	\item \textbf{Tạo cơ sở dữ liệu và bảng biểu}: Tạo cơ sở dữ liệu \code{dht\_sensor\_db} và các bảng \code{sensor\_readings}, \code{device\_settings}, \code{users} (cho chức năng đăng nhập) theo cấu trúc đã thiết kế.
	\item \textbf{Triển khai Backend và Frontend}: Đặt các thư mục \file{TH\_DO\_AN/} và \file{Login/} vào thư mục \code{htdocs} của \code{XAMPP}. Đảm bảo các file \file{config\_cors.php} cũng được đặt đúng vị trí nếu có.
	\item \textbf{Lắp ráp phần cứng}: Kết nối cảm biến \keyword{DHT11} và \keyword{LED} với các chân \code{GPIO} tương ứng trên \file{ESP32}.
	\item \textbf{Nạp Firmware}: Nạp \keyword{firmware} đã biên dịch vào \file{ESP32} bằng \code{Arduino IDE}.
	\item \textbf{Cấu hình mạng Wi-Fi}: Cấu hình \file{ESP32} để kết nối với mạng \keyword{Wi-Fi} cục bộ (ví dụ: điểm phát sóng từ điện thoại \code{Redmi Note 12 Pro}).
	\item \textbf{Kiểm thử và hiệu chuẩn}: Kiểm tra toàn bộ luồng dữ liệu, chức năng hiển thị, cảnh báo và điều khiển, cũng như chức năng đăng nhập/đăng ký để đảm bảo hệ thống hoạt động chính xác.
\end{itemize} 
	%% Chương 3

\chapter{KẾT QUẢ VÀ ĐÁNH GIÁ HỆ THỐNG} % Tên của chương

\label{Chapter3} % Để trích dẫn chương này ở chỗ nào đó trong bài, hãy sử dụng lệnh \ref{Chapter3} 

%----------------------------------------------------------------------------------------

\section{Dữ liệu thu thập và hiển thị}
Hệ thống đã được triển khai và hoạt động liên tục, thu thập dữ liệu nhiệt độ và độ ẩm từ cảm biến \keyword{DHT11} tại môi trường trong nhà. Dữ liệu này được truyền về máy chủ cục bộ và hiển thị thông qua giao diện web trực quan.

\subsection{Hiển thị dữ liệu thời gian thực}
Giao diện "Dữ liệu Hiện Tại" cung cấp một cái nhìn nhanh chóng về điều kiện môi trường. Nhiệt độ và độ ẩm được hiển thị rõ ràng dưới dạng các thẻ lớn, cùng với thời gian cập nhật gần nhất.

\begin{figure}[hbtp]
	\centering
	\includegraphics[width=0.9\textwidth]{Figures/current_data_interface.png} % Thay thế bằng ảnh chụp màn hình thực tế của bạn
	\caption{Giao diện hiển thị dữ liệu hiện tại}
	\label{fig:current_data}
\end{figure}
Hình~\ref{fig:current_data} minh họa giao diện hiển thị dữ liệu hiện tại.

\subsection{Lịch sử dữ liệu và biểu đồ}
Tính năng "Lịch sử Dữ liệu" cho phép người dùng xem biểu đồ biến động của nhiệt độ và độ ẩm theo thời gian. Người dùng có thể chọn khoảng thời gian cụ thể (ví dụ: 1 giờ) để phân tích xu hướng. Biểu đồ tương tác giúp dễ dàng nhận diện các thay đổi.

\begin{figure}[hbtp]
	\centering
	\includegraphics[width=0.9\textwidth]{Figures/history_data_chart.png} % Thay thế bằng ảnh chụp màn hình thực tế của bạn
	\caption{Biểu đồ lịch sử dữ liệu nhiệt độ và độ ẩm}
	\label{fig:history_data}
\end{figure}
Hình~\ref{fig:history_data} thể hiện biểu đồ lịch sử dữ liệu. Ngoài ra, hệ thống hỗ trợ xuất dữ liệu lịch sử ra file \file{CSV} để tiện cho việc phân tích offline.

\section{Tính năng cảnh báo và điều khiển}
\subsection{Thiết lập ngưỡng và trạng thái cảnh báo}
Hệ thống cung cấp trang "Cảnh Báo" nơi người dùng có thể thiết lập các ngưỡng an toàn cho nhiệt độ và độ ẩm. Khi dữ liệu đo được vượt quá các ngưỡng này, giao diện web sẽ tự động hiển thị trạng thái "Vượt ngưỡng" và kích hoạt hiệu ứng nhấp nháy đỏ trên các thẻ hiển thị dữ liệu để cảnh báo trực quan.

\begin{figure}[hbtp]
	\centering
	\includegraphics[width=0.9\textwidth]{Figures/alert_settings_interface.png} % Thay thế bằng ảnh chụp màn hình thực tế của bạn
	\caption{Giao diện thiết lập ngưỡng và trạng thái cảnh báo}
	\label{fig:alert_settings}
\end{figure}
Giao diện thiết lập cảnh báo được thể hiện trong Hình~\ref{fig:alert_settings}.

\subsection{Điều khiển bật/tắt cảm biến}
Một tính năng tiện lợi của hệ thống là khả năng điều khiển bật/tắt việc thu thập dữ liệu của cảm biến trực tiếp từ giao diện web. Nút "Đang đo" (hoặc tương tự) cho phép người dùng tạm dừng hoặc tiếp tục hoạt động của cảm biến \file{ESP32}, giúp tiết kiệm năng lượng hoặc khi không cần giám sát.

\section{Đánh giá hiệu quả hoạt động}
\subsection{Ưu điểm}
\begin{itemize}
	\item \textbf{Chi phí thấp và dễ triển khai}: Sử dụng các linh kiện phổ biến như \keyword{DHT11} và \file{ESP32}, cùng nền tảng \code{XAMPP} miễn phí, giúp giảm đáng kể chi phí tổng thể của hệ thống.
	\item \textbf{Giao diện trực quan và thân thiện}: Giao diện web được thiết kế đơn giản, dễ hiểu, cho phép người dùng cuối dễ dàng theo dõi và tương tác mà không cần kiến thức chuyên sâu.
	\item \textbf{Giám sát thời gian thực và lịch sử}: Cung cấp cái nhìn toàn diện về môi trường hiện tại và khả năng phân tích xu hướng dài hạn.
	\item \textbf{Tính năng cảnh báo và điều khiển}: Nâng cao khả năng phản ứng và quản lý môi trường, giúp người dùng kịp thời nhận biết và xử lý các vấn đề.
\end{itemize}

\subsection{Hạn chế và thách thức đã gặp phải}
Trong quá trình triển khai và vận hành hệ thống, một số thách thức đã được ghi nhận:
\begin{itemize}
	\item \textbf{Hoạt động không liên tục và kết nối WiFi không ổn định}: Hệ thống gặp khó khăn trong việc duy trì hoạt động liên tục trong thời gian dài và kết nối \keyword{Wi-Fi} giữa \file{ESP32} và máy chủ \keyword{localhost} đôi khi không ổn định. Điều này có thể do các yếu tố như nhiễu sóng, phạm vi phủ sóng hạn chế hoặc vấn đề về quản lý năng lượng trên \file{ESP32}.
	\item \textbf{Độ chính xác và độ bền của cảm biến}: \keyword{DHT11} là cảm biến cơ bản, có thể có giới hạn về độ chính xác và độ bền trong các môi trường khắc nghiệt hơn hoặc yêu cầu giám sát chặt chẽ.
	\item \textbf{Phạm vi giám sát}: Hiện tại hệ thống chỉ hoạt động trong mạng cục bộ (\keyword{localhost}), hạn chế khả năng truy cập dữ liệu từ xa.
	\item \textbf{Xử lý dữ liệu nâng cao}: Hệ thống chưa tích hợp các thuật toán phức tạp hơn để lọc nhiễu dữ liệu hoặc phân tích dự đoán chuyên sâu.
\end{itemize}
	%% Chương 4

\chapter{KẾT LUẬN VÀ ĐỊNH HƯỚNG PHÁT TRIỂN} % Tên của chương

\label{Chapter4} % Để trích dẫn chương này ở chỗ nào đó trong bài, hãy sử dụng lệnh \ref{Chapter4} 

%----------------------------------------------------------------------------------------

\section{Kết luận}
Báo cáo này đã trình bày chi tiết quá trình thiết kế và triển khai một hệ thống giám sát nhiệt độ và độ ẩm thông minh sử dụng \file{ESP32}, cảm biến \keyword{DHT11}, cùng với một \keyword{backend} \code{PHP/MySQL} và \keyword{frontend} \code{HTML/CSS/JavaScript}. Hệ thống đã đạt được các mục tiêu đề ra ban đầu, bao gồm:
\begin{itemize}
	\item Thu thập và hiển thị dữ liệu nhiệt độ và độ ẩm theo thời gian thực một cách hiệu quả.
	\item Cung cấp giao diện web trực quan để theo dõi dữ liệu hiện tại, lịch sử và thiết lập cảnh báo.
	\item Hỗ trợ chức năng điều khiển bật/tắt cảm biến từ xa.
\end{itemize}
Mặc dù hệ thống vẫn còn một số hạn chế về độ ổn định kết nối và khả năng hoạt động liên tục, nhưng nó đã chứng minh được tính khả thi và tiềm năng ứng dụng trong việc giám sát chất lượng môi trường trong nhà.

\section{Định hướng phát triển và cải tiến trong tương lai}
Dựa trên những kết quả đạt được và các thách thức đã nhận diện, hệ thống có thể được phát triển và cải tiến theo các hướng sau:
\begin{itemize}
	\item \textbf{Mở rộng thông số giám sát}: Tích hợp thêm các loại cảm biến môi trường khác như cảm biến bụi (\keyword{PM2.5}, \keyword{PM10}), cảm biến khí \keyword{CO2}, cảm biến khí độc, tiếng ồn, ánh sáng, hoặc thậm chí là cảm biến còi để có cái nhìn toàn diện hơn về môi trường.
	\item \textbf{Nâng cấp giao diện web và tính năng}: Phát triển thêm các chức năng như quản lý nhiều thiết bị từ một giao diện, phân quyền người dùng (admin, người dùng thường), tạo báo cáo tùy chỉnh, hoặc tích hợp bản đồ hiển thị vị trí các cảm biến.
	\item \textbf{Tích hợp nền tảng đám mây}: Thay vì chỉ lưu trữ trên \keyword{localhost}, dữ liệu nên được đẩy lên các dịch vụ \keyword{IoT} \keyword{cloud} (ví dụ: \code{ThingSpeak}, \code{Firebase}, \code{AWS IoT}, \code{Google Cloud IoT}). Điều này không chỉ giải quyết vấn đề về bộ nhớ mà còn cho phép truy cập và phân tích dữ liệu từ mọi nơi, mọi lúc.
	\item \textbf{Cải thiện độ ổn định và quản lý năng lượng}: Nghiên cứu các giải pháp phần cứng và \keyword{firmware} để tối ưu hóa việc tiêu thụ năng lượng của \file{ESP32}, kéo dài thời gian hoạt động liên tục. Đồng thời, tìm cách cải thiện độ ổn định của kết nối \keyword{Wi-Fi} hoặc xem xét các công nghệ truyền thông khác phù hợp hơn (\code{LoRa}, \code{NB-IoT}) cho các ứng dụng cần tầm xa hoặc tiêu thụ năng lượng cực thấp.
	\item \textbf{Phân tích dữ liệu nâng cao}: Triển khai các thuật toán xử lý tín hiệu để lọc nhiễu từ cảm biến, và áp dụng các phương pháp phân tích dữ liệu (ví dụ: học máy) để dự đoán xu hướng ô nhiễm hoặc phát hiện các sự kiện bất thường một cách tự động.
\end{itemize}
Những định hướng này sẽ giúp hệ thống trở nên mạnh mẽ, đáng tin cậy và có khả năng ứng dụng rộng rãi hơn trong tương lai. 
	%% Chương 5

\chapter{KẾT LUẬN VÀ KHUYẾN NGHỊ} % Tên của chương

\label{Chapter5} % Để trích dẫn chương này ở chỗ nào đó trong bài, hãy sử dụng lệnh \ref{Chapter5} 

%----------------------------------------------------------------------------------------

\section{Kết luận}
Báo cáo này đã trình bày quá trình nghiên cứu, thiết kế, triển khai và đánh giá một hệ thống giám sát nhiệt độ và độ ẩm thông minh dựa trên nền tảng \keyword{IoT}. Hệ thống đã chứng minh được khả năng hoạt động ổn định trong việc thu thập và hiển thị dữ liệu môi trường theo thời gian thực, cung cấp giao diện trực quan cho người dùng, và tích hợp các tính năng quan trọng như thiết lập cảnh báo ngưỡng và điều khiển bật/tắt cảm biến. Các mục tiêu ban đầu về việc xây dựng một giải pháp giám sát hiệu quả và chi phí thấp đã được đáp ứng.

\section{Khuyến nghị}
Để tối ưu hóa và mở rộng tiềm năng của hệ thống, chúng tôi đề xuất các khuyến nghị sau:
\begin{itemize}
	\item \textbf{Đối với các nghiên cứu tiếp theo}: Nên tập trung vào việc khắc phục triệt để vấn đề kết nối \keyword{Wi-Fi} không ổn định và tối ưu hóa quản lý năng lượng của thiết bị biên. Đồng thời, việc khám phá các loại cảm biến mới với độ chính xác và độ bền cao hơn sẽ cải thiện chất lượng dữ liệu.
	\item \textbf{Đối với ứng dụng thực tế}: Cân nhắc tích hợp hệ thống với các nền tảng \keyword{IoT} \keyword{cloud} để nâng cao khả năng truy cập từ xa và quản lý dữ liệu tập trung. Việc phát triển thêm các module cảm biến cho các thông số ô nhiễm khác (\keyword{PM2.5}, \keyword{CO2}, khí độc) là cần thiết để hệ thống trở nên toàn diện hơn.
	\item \textbf{Đối với phát triển phần mềm}: Đầu tư vào việc phát triển các thuật toán phân tích dữ liệu nâng cao (lọc nhiễu, dự đoán xu hướng) và cải thiện giao diện người dùng để hệ thống trở nên thông minh và thân thiện hơn nữa.
\end{itemize}
Hệ thống giám sát nhiệt độ và độ ẩm này là một bước khởi đầu đầy hứa hẹn, mở ra nhiều cơ hội để phát triển thành một giải pháp giám sát môi trường toàn diện và bền vững. 
	
	
	%----------------------------------------------------------------------------------------
	%	(KHÔNG CHỈNH SỬA PHẦN NÀY)
	%
	%	PHẦN 13: TÀI LIỆU THAM KHẢO
	%----------------------------------------------------------------------------------------
	
	\begin{spacing}{1.15}
		\printbibliography[heading=bibintoc, title=Tài liệu tham khảo] % In ra tài liệu tham khảo
	\end{spacing}
	
	
	%----------------------------------------------------------------------------------------
	%	PHẦN 14: PHỤ LỤC (THESIS CONTENT - APPENDICES)
	%----------------------------------------------------------------------------------------
	
	\appendix % Nói với LaTeX rằng những chương về sau được tính là phụ lục
	
	% Hãy thêm những phụ lục (appendix) của báo cáo vào thư mục Appendices
	% Hãy bỏ chú thích những dòng nếu bạn đã bổ sung những phụ lục vào
	
	\include{Appendices/AppendixA}
	% Phụ lục B

\chapter{LIỆT KÊ SOURCE CODE} % Tên của phụ lục

\label{AppendixB} % Để trích dẫn chương này ở chỗ nào đó trong bài, hãy sử dụng lệnh \ref{AppendixB} 

%----------------------------------------------------------------------------------------

\section{Mã nguồn Firmware ESP32}
Mã nguồn \keyword{firmware} được nạp vào \file{ESP32} để thực hiện các chức năng thu thập dữ liệu, điều khiển \keyword{LED} và truyền thông.
\lstinputlisting[style=codeC]{"Code/test-do-an-led.ino"} % Thay "esp32_firmware.ino" bằng tên file thực tế của bạn

\section{Mã nguồn Backend (PHP)}
Các file \code{PHP} xử lý các yêu cầu từ \file{ESP32} và giao diện web, quản lý cơ sở dữ liệu \code{MySQL}.
\subsection{Các file chính trong TH\_DO\_AN/}
\lstinputlisting[style=codePHP]{"Code/TH_DO_AN/post_data.php"} % Thay bằng tên file thực tế của bạn
\lstinputlisting[style=codePHP]{"Code/TH_DO_AN/get_current_data.php"} % Thay bằng tên file thực tế của bạn
\lstinputlisting[style=codePHP]{"Code/TH_DO_AN/get_history_data.php"} % Thay bằng tên file thực tế của bạn
\lstinputlisting[style=codePHP]{"Code/TH_DO_AN/test_db.php"} % Thay bằng tên file thực tế của bạn
\lstinputlisting[style=codePHP]{"Code/TH_DO_AN/db_connect.php"} % Thay bằng tên file thực tế của bạn
\lstinputlisting[style=codePHP]{"Code/TH_DO_AN/app_config.php"} % Thay bằng tên file thực tế của bạn

\subsection{Các file API trong TH\_DO\_AN/api/}
\lstinputlisting[style=codePHP]{"Code/TH_DO_AN/api/get_devices.php"} % Thay bằng tên file thực tế của bạn
\lstinputlisting[style=codePHP]{"Code/TH_DO_AN/api/export_data_csv.php"} % Thay bằng tên file thực tế của bạn
\lstinputlisting[style=codePHP]{"Code/TH_DO_AN/api/save_alert_thresholds.php"} % Thay bằng tên file thực tế của bạn
\lstinputlisting[style=codePHP]{"Code/TH_DO_AN/api/save_email_settings.php"} % Thay bằng tên file thực tế của bạn
\lstinputlisting[style=codePHP]{"Code/TH_DO_AN/api/save_update_frequency.php"} % Thay bằng tên file thực tế của bạn
\lstinputlisting[style=codePHP]{"Code/TH_DO_AN/api/get_alert_thresholds.php"} % Thay bằng tên file thực tế của bạn
\lstinputlisting[style=codePHP]{"Code/TH_DO_AN/api/get_dht_status.php"} % Thay bằng tên file thực tế của bạn
\lstinputlisting[style=codePHP]{"Code/TH_DO_AN/api/toggle_dht_status.php"} % Thay bằng tên file thực tế của bạn

\subsection{Các file API người dùng trong Login/api/users/}
\lstinputlisting[style=codePHP]{"Code/Login/api/users/check_login.php"} % Thay bằng tên file thực tế của bạn
\lstinputlisting[style=codePHP]{"Code/Login/api/users/login.php"} % Thay bằng tên file thực tế của bạn
\lstinputlisting[style=codePHP]{"Code/Login/api/users/logout.php"} % Thay bằng tên file thực tế của bạn
\lstinputlisting[style=codePHP]{"Code/Login/api/users/register.php"} % Thay bằng tên file thực tế của bạn

\section{Mã nguồn Frontend (HTML/CSS/JavaScript)}
Các file xây dựng giao diện người dùng web.
\subsection{Các file chính trong TH\_DO\_AN/}
\lstinputlisting[style=codeHTML]{"Code/TH_DO_AN/index.php"} % Nếu index.php chứa HTML và PHP, sử dụng codeHTML
\lstinputlisting[style=plaintext]{"Code/TH_DO_AN/main.css"} % Style plaintext cho CSS
\lstinputlisting[style=codeJavaScript]{"Code/TH_DO_AN/main.js"} % Style codeJavaScript
\lstinputlisting[style=codeHTML]{"Code/TH_DO_AN/dashboard_template.html"} % Style codeHTML

\subsection{Các file trong Login/}
\lstinputlisting[style=codeHTML]{"Code/Login/login_register.html"} % Style codeHTML
\lstinputlisting[style=plaintext]{"Code/Login/auth_styles.css"} % Style plaintext
\lstinputlisting[style=codeJavaScript]{"Code/Login/auth_script.js"} % Style codeJavaScript

% Bạn có thể thêm các file code ví dụ khác nếu cần
% \section{Ví dụ liệt kê code ngôn ngữ C/C++}
% Code tính khoảng thời gian giữa hai thời điểm cho trước.
% \lstinputlisting[style=codeC]{"Code/TimeDiff.cpp"} 

% \section{Ví dụ liệt kê code ngôn ngữ Python}
% Code tính mã hash của một file.
% \lstinputlisting[style=codePython]{"Code/Hash.py"}
	%\include{Appendices/AppendixC}
	
	%----------------------------------------------------------------------------------------
	
\end{document}