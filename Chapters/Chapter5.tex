% Chương 5

\chapter{KẾT LUẬN VÀ KHUYẾN NGHỊ} % Tên của chương

\label{Chapter5} % Để trích dẫn chương này ở chỗ nào đó trong bài, hãy sử dụng lệnh \ref{Chapter5} 

%----------------------------------------------------------------------------------------

\section{Kết luận}
Báo cáo này đã trình bày quá trình nghiên cứu, thiết kế, triển khai và đánh giá một hệ thống giám sát nhiệt độ và độ ẩm thông minh dựa trên nền tảng \keyword{IoT}. Hệ thống đã chứng minh được khả năng hoạt động ổn định trong việc thu thập và hiển thị dữ liệu môi trường theo thời gian thực, cung cấp giao diện trực quan cho người dùng, và tích hợp các tính năng quan trọng như thiết lập cảnh báo ngưỡng và điều khiển bật/tắt cảm biến. Các mục tiêu ban đầu về việc xây dựng một giải pháp giám sát hiệu quả và chi phí thấp đã được đáp ứng.

\section{Khuyến nghị}
Để tối ưu hóa và mở rộng tiềm năng của hệ thống, chúng tôi đề xuất các khuyến nghị sau:
\begin{itemize}
	\item \textbf{Đối với các nghiên cứu tiếp theo}: Nên tập trung vào việc khắc phục triệt để vấn đề kết nối \keyword{Wi-Fi} không ổn định và tối ưu hóa quản lý năng lượng của thiết bị biên. Đồng thời, việc khám phá các loại cảm biến mới với độ chính xác và độ bền cao hơn sẽ cải thiện chất lượng dữ liệu.
	\item \textbf{Đối với ứng dụng thực tế}: Cân nhắc tích hợp hệ thống với các nền tảng \keyword{IoT} \keyword{cloud} để nâng cao khả năng truy cập từ xa và quản lý dữ liệu tập trung. Việc phát triển thêm các module cảm biến cho các thông số ô nhiễm khác (\keyword{PM2.5}, \keyword{CO2}, khí độc) là cần thiết để hệ thống trở nên toàn diện hơn.
	\item \textbf{Đối với phát triển phần mềm}: Đầu tư vào việc phát triển các thuật toán phân tích dữ liệu nâng cao (lọc nhiễu, dự đoán xu hướng) và cải thiện giao diện người dùng để hệ thống trở nên thông minh và thân thiện hơn nữa.
\end{itemize}
Hệ thống giám sát nhiệt độ và độ ẩm này là một bước khởi đầu đầy hứa hẹn, mở ra nhiều cơ hội để phát triển thành một giải pháp giám sát môi trường toàn diện và bền vững.