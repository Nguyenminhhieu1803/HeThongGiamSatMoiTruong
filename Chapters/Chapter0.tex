% MỞ ĐẦU (LỜI MỞ ĐẦU - Chương 0)

\chapter*{MỞ ĐẦU} % Tên của chương
\addcontentsline{toc}{chapter}{MỞ ĐẦU} % Thêm tên chương vào mục lục

\label{Chapter0} % Để trích dẫn chương này ở chỗ nào đó trong bài, hãy sử dụng lệnh \ref{Chapter0} 

%----------------------------------------------------------------------------------------

Phần "MỞ ĐẦU" của báo cáo này trình bày tổng quan về hệ thống giám sát nhiệt độ và độ ẩm thông minh, một giải pháp được phát triển nhằm theo dõi chất lượng không khí trong môi trường nhà ở và văn phòng. Hệ thống này được thiết kế với mục tiêu cung cấp dữ liệu môi trường theo thời gian thực, hỗ trợ người dùng theo dõi lịch sử dữ liệu, và đưa ra cảnh báo kịp thời khi các thông số vượt ngưỡng cho phép.

Hệ thống nổi bật với khả năng triển khai đơn giản, chi phí thấp, và giao diện web thân thiện, dễ sử dụng, phù hợp cho các ứng dụng giám sát chất lượng không khí trong nhà. Báo cáo này sẽ đi sâu vào các khía cạnh từ bối cảnh và sự cần thiết, mục tiêu, đến thiết kế phần cứng và phần mềm, quy trình triển khai, cũng như kết quả hoạt động, các thách thức đã gặp phải và hướng phát triển trong tương lai.

Bố cục của báo cáo được tổ chức như sau:
\begin{itemize}
	\item \textbf{Chương 1: Tổng quan về hệ thống giám sát nhiệt độ và độ ẩm} trình bày bối cảnh, sự cần thiết và các thành phần chính của hệ thống.
	\item \textbf{Chương 2: Thiết kế và triển khai hệ thống} mô tả chi tiết kiến trúc phần cứng và phần mềm, cùng với quy trình xây dựng và lắp đặt hệ thống.
	\item \textbf{Chương 3: Kết quả và đánh giá hệ thống} trình bày các dữ liệu thu thập được, phân tích hiệu quả hoạt động và các tính năng cảnh báo, điều khiển.
	\item \textbf{Chương 4: Kết luận và định hướng phát triển} tóm tắt những thành quả chính và đề xuất các hướng cải tiến, mở rộng hệ thống trong tương lai.
	\item \textbf{Phụ lục} bao gồm mã nguồn chi tiết của các module hệ thống.
\end{itemize}